\documentclass[a4paper,12pt]{report}

% Packages
\usepackage[utf8]{inputenc}
\usepackage[T1]{fontenc}
\usepackage[english]{babel}
\usepackage{amsmath}
\usepackage{amsfonts}
\usepackage{amssymb}
\usepackage{graphicx}
\usepackage{subcaption}
\usepackage{hyperref}
\usepackage{booktabs}
\usepackage{multirow}
\usepackage{array}

% Title page
\title{Exploring Transformer-Based Models for Named Entity Recognition in Ukrainian}
\author{Your Name}
\date{Month Year}

\begin{document}

% Title page
\begin{titlepage}
\centering
\vspace*{2cm}
{\Large\textbf{Exploring Transformer-Based Models for Named Entity Recognition in Ukrainian}}\\
\vspace{1cm}
{\large\textbf{Master's Thesis}}\\
\vspace{2cm}
{\large\textbf{Timo Junolainen}}\\
\vspace{1cm}
{\large\textbf{Februrary 2023}}\\
\vfill
{\large\textbf{Department of Computer Science}}\\
{\large\textbf{University of Turku}}\\
\end{titlepage}

% Abstract
\section*{Abstract}
\addcontentsline{toc}{chapter}{Abstract}
Named entity recognition (NER)\cite{wiki:Named-entity_recognition} is an important task in natural language processing, consisting of identifying and classifying named entities in text. With the recent success of transformer-based models for NER in high-resource languages, this thesis explores the effectiveness of these models for NER in a low-resource\cite{lowresource} language: Ukrainian\cite{wiki:Ukrainian_language}. The research questions addressed in this thesis include the effectiveness of transformer-based models for NER in Ukrainian, the impact of different pre-training and fine-tuning strategies on model performance, and a comparison of the results to existing state-of-the-art methods for NER in Ukrainian. The objectives of this thesis are to develop transformer-based models for NER in Ukrainian, evaluate their performance on standard benchmarks, analyze the impact of different pre-training and fine-tuning strategies, and compare the results to existing state-of-the-art methods. The experimental results show that transformer-based models can achieve competitive results on NER in Ukrainian with appropriate pre-training and fine-tuning strategies. This thesis contributes to the development of NER models for low-resource languages and provides insights into the effectiveness of transformer-based models for NER in such languages.

% Table of contents
\tableofcontents

% Introduction
\chapter{Introduction}
\section{Motivation behind the work }

Named entity recognition (NER)\cite{wiki:Named-entity_recognition} is one of the primary tasks in natural language processing, and consists of classifying named entities in text, such as people, organizations, and locations. 
\par 
Last years, with use of deep learning methods, with transformers, state-of-the-arts results have been achieved with English, Spanish and Chinese languages (ref ?) 
\par 
It must be observed, that English, Spanish and Chinese, are high-resource languages,  meaning essentially that a lot of computating power went into finetuning models for English, Spanish and Chinese. 
\par
However, there are much so called “low-resource” languages, as for example: 
\par
% Low resource languages
\begin{enumerate}
    \item     Basque: spoken in the Basque Country, an autonomous region of northern Spain and southwestern France. 
    \item     Belarusian: spoken in Belarus, a country in Eastern Europe. 
    \item     Fijian: spoken in Fiji, an island country in the South Pacific. 
    \item     Irish: spoken primarily in Ireland and Northern Ireland. 
    \item     Kyrgyz: spoken in Kyrgyzstan, a country in Central Asia. 
    \item     Luxembourgish: spoken in Luxembourg, a small country in Europe. 
    \item     Maltese: spoken in Malta, a small island nation in the Mediterranean. 
    \item     Samoan: spoken in Samoa, an island nation in the South Pacific. 
    \item     Scottish Gaelic: spoken primarily in Scotland. 
    \item     Yoruba: spoken in Nigeria and other West African countries. 
\end{enumerate}
% Literature review
\chapter{Literature Review}
This chapter reviews the literature on NER, with a focus on transformer-based models for NER and existing work on

% Methodology
\chapter{Methodology}
\chapter{Results}
\chapter{Conclusions}

\bibliographystyle{plain}
\bibliography{mt.bib}
    
    
\end{document}