\documentclass[a4paper,12pt]{report}

% Packages
\usepackage[utf8]{inputenc}
\usepackage[T1]{fontenc}
\usepackage[english]{babel}
\usepackage{amsmath}
\usepackage{amsfonts}
\usepackage{amssymb}
\usepackage{graphicx}
\usepackage{subcaption}
\usepackage{hyperref}
\usepackage{booktabs}
\usepackage{multirow}
\usepackage{array}


% Title page
\title{Exploring Transformer-Based Models for Named Entity Recognition in Ukrainian}
\author{Your Name}
\date{Month Year}

\begin{document}

% Title page
\begin{titlepage}
\centering
\vspace*{2cm}
{\Large\textbf{Exploring Transformer-Based Models for Named Entity Recognition in Ukrainian}}\\
\vspace{1cm}
{\large\textbf{Master's Thesis}}\\
\vspace{2cm}
{\large\textbf{Timo Junolainen}}\\
\vspace{1cm}
{\large\textbf{Februrary 2023}}\\
\vfill
{\large\textbf{Department of Computer Science}}\\
{\large\textbf{University of Turku}}\\
\end{titlepage}

% Abstract
\section*{Abstract}
\addcontentsline{toc}{chapter}{Abstract}
Named entity recognition (NER)\cite{wiki:Named-entity_recognition} is an important task in natural language processing, consisting of identifying and classifying named entities in text. With the recent success of transformer-based models for NER in high-resource languages, this thesis explores the effectiveness of these models for NER in a low-resource\cite{lowresource} language: Ukrainian\cite{wiki:Ukrainian_language}. The research questions addressed in this thesis include the effectiveness of transformer-based models for NER in Ukrainian, the impact of different pre-training and fine-tuning strategies on model performance, and a comparison of the results to existing state-of-the-art methods for NER in Ukrainian. The objectives of this thesis are to develop transformer-based models for NER in Ukrainian, evaluate their performance on standard benchmarks, analyze the impact of different pre-training and fine-tuning strategies, and compare the results to existing state-of-the-art methods. The experimental results show that transformer-based models can achieve competitive results on NER in Ukrainian with appropriate pre-training and fine-tuning strategies. This thesis contributes to the development of NER models for low-resource languages and provides insights into the effectiveness of transformer-based models for NER in such languages.

% Table of contents
\tableofcontents

% Introduction
\chapter{Introduction}
\section{Motivation behind the work }

Named entity recognition (NER)\cite{wiki:Named-entity_recognition} is one of the primary tasks in natural language processing, and consists of classifying named entities in text, such as people, organizations, and locations. 
\par 
Last years, with use of deep learning methods, with transformers, state-of-the-arts results have been achieved with English, Spanish and Chinese languages (ref ?) 
\par 
It must be observed, that English, Spanish and Chinese, are high-resource languages,  meaning essentially that a lot of computating power went into finetuning models for English, Spanish and Chinese. 
\par
However, there are much so called “low-resource” languages, as for example: 
\par
% Low resource languages
\begin{enumerate}
    \item     Basque: spoken in the Basque Country, an autonomous region of northern Spain and southwestern France. 
    \item     Belarusian: spoken in Belarus, a country in Eastern Europe. 
    \item     Fijian: spoken in Fiji, an island country in the South Pacific. 
    \item     Irish: spoken primarily in Ireland and Northern Ireland. 
    \item     Kyrgyz: spoken in Kyrgyzstan, a country in Central Asia. 
    \item     Luxembourgish: spoken in Luxembourg, a small country in Europe. 
    \item     Maltese: spoken in Malta, a small island nation in the Mediterranean. 
    \item     Samoan: spoken in Samoa, an island nation in the South Pacific. 
    \item     Scottish Gaelic: spoken primarily in Scotland. 
    \item     Yoruba: spoken in Nigeria and other West African countries. 
\end{enumerate}
\par
Effectiveness of transformer-based models in context of Ukrainian language specifically, is going to be explored in this research. Ukrainian is second most spoken language in Eastern Europe, but there is lack of high-quality annotated data and linguistic resources for Ukrainian language.
\par
Impact of different pre-training and finetuning strategies with transformer models will be evaluated with standard metrics and benchmarks.
%1.2
\section{Questions to research}
List of research questions
\begin{enumerate}
    \item How effective are transformer-based models for NER in Ukrainian, a low-resource language
    \item How effectiveness of performance is impacted by different pretraining and finetuning strategies in Ukrainian language
    \item How transfomer based models compare to current state-of-the-art methods for NER in Ukrainian language
\end{enumerate}
Objectives of this research are
\begin{enumerate}
    \item Development of transformer based models using variety of pretrained and finetuned methods
    \item Evaluation of models using standart NER benchmarks
    \item Analyze impact of different pretraining and finetuning strategies on the performance of the models
    \item Compare results to current best methods
\end{enumerate}

% 1.3
\section{Overview of this works report structure}
This paper is build as:
\begin{enumerate}
    \item The \autoref{cap:literature} provides review of literature on NER with a focus on transformer-based models in lowresource languages
    \item The \autoref{cap:methodology} describes methodology used in this work, including dataset and its preprocessing steps, as well transfomer models being used
    \item The \autoref{cap:results} presentes results of experiments and analysis of impact of different strategies on the perfomance of transfomer-based models.
    \item The \autoref{cap:conclusions} provides findings and contributions of the work, discusses current limitations and future research
\end{enumerate}

% 2 Literature review
\chapter{Literature Review}\label{cap:literature}
This chapter reviews the literature on NER, with a focus on transformer-based models for NER and existing work on.
% 2.1
\section{Brief review, with history of development in the field}
Natural Language Processing (NLP) has a section called named entity recognition (NER) that deals with extracting names, places, organizations, and dates from text. The history of NER development is extensive and spans several decades.
\par
Rule-based techniques were employed in the 1980s to locate named items in text. These methods entailed manually establishing rules to recognize particular word patterns or word combinations that matched designated things. Unfortunately, the ability of these rule-based systems to deal with ambiguity and variety in language use was frequently constrained.
\par
Statistical techniques started to take hold as a fresh approach to NER in the 1990s. With the use of these techniques, named entities might be automatically identified based on statistical patterns by training machine learning models on massive datasets of annotated text. The Maximum Entropy Markov Model was one of the first statistical models for NER (MEMM).
\par
Conditional Random Fields (CRFs) sprang to prominence as a NER strategy in the early 2000s. In order to perform better on NER tasks, CRFs are a type of statistical model that can take into account the context and relationships between nearby words in a phrase.
\par
Deep learning techniques have also recently been used to NER, notably when using recurrent neural networks (RNNs) and convolutional neural networks as neural network designs (CNNs). Results from these models have been encouraging, especially for tasks involving big and complicated datasets.
\par 
Transformer-based models have recently revolutionized NLP and vastly enhanced NER system performance. Examples of these models include the BERT and GPT series. Transformers are a particular sort of neural network design that are very effective for NLP tasks because they can process sequences of tokens, such words or characters, in parallel.
\par
For instance, BERT is a pre-trained transformer-based model that has been improved on a number of NER-related downstream NLP tasks. BERT has attained state-of-the-art performance on various NER benchmarks by pre-training on massive volumes of text data and then fine-tuning on certain tasks.
\par
Transformers have also facilitated the creation of novel NER strategies, such as combining pre-trained and task-specific transformers to enhance performance on NER tasks that are specialized to a given domain.
\par
In conclusion, transformers have become an effective tool for creating NER systems, allowing for more accurate and efficient processing of text input and fostering further advancement in the area.
%\section{Existing NER methods, including Machine Learning approach}
%drop this?
% 2.2
\section{Discussion about current advances for NER field}
Transformer-based models have become a potent NER strategy in recent years, generating cutting-edge results on a variety of benchmark datasets. These models have been demonstrated to be very successful at modeling natural language tasks because they make use of the self-attention mechanism, which enables them to capture long-range relationships in the input sequence.
\par
BERT is one of the most popular transformer-based NER models (Bidirectional Encoder Representations from Transformers). BERT is a pre-trained model that has been optimized for NER as well as other NER-related NLP tasks. BERT has been demonstrated to attain state-of-the-art performance on a variety of languages, including English, Chinese, and German, when fine-tuned on a particular NER dataset.
\par
For NER, other transformer-based models like RoBERTa, ALBERT, and ELECTRA have also been created. These models expand upon the BERT architecture and enhance it in a number of ways, such as by using greater training data, more complex pre-training objectives, or more effective training techniques.
\par
The substantial amount of computer resources needed for training and inference is one of the difficulties of employing transformer-based models for NER. Recent studies have, however, looked into techniques to lower these models' computing costs, such as knowledge distillation, pruning, or quantization.
\par
\section{Review of related NER works in the field of low-resource languages}
The creation of NER models for low-resource languages is a tough task, as the amount of annotated data is generally limited. Nonetheless, a number of recent studies have investigated methods to solve this issue and enhance the functionality of NER models for low-resource languages.
\par
One strategy is to use multilingual pre-trained models that have been trained on a variety of languages, such as XLM-R or mBERT, which can be adjusted on a low-resource language with minimal annotated data. It has been demonstrated that this strategy works for a number of low-resource languages, including Vietnamese, Swahili, and Bengali.
\par
A different strategy is to apply cross-lingual transfer learning, in which information from a language with high resources is transferred to a language with low resources. For instance, Zhang et al. (2021)\cite{crosslanguage} introduced a cross-lingual transfer learning system that converts information from English to African languages with limited resources while obtaining cutting-edge performance on numerous benchmark datasets.
\par
Another strategy that has been investigated for NER in low-resource languages is active learning. In order to increase the performance of the model while using less annotated data, active learning entails choosing the most instructive data points to annotate. Al-Rfou et al. (2013), for instance, used active learning to create an Arabic NER system and achieved competitive performance with a minimal amount of annotated data.
\par
Overall, there have been several promising works that have explored approaches to improve the performance of NER models for low-resource languages. However, there is still a need for further research in this area to address the challenges of limited annotated data and to develop more effective approaches for low-resource languages.

% Methodology
\chapter{Methodology}\label{cap:methodology}
\section{Dataset description and preprocessing}
** About dataset **
\section{Overview of transformer models, including strategies of pretraining and finetunning}
In tasks involving named entity recognition in natural language processing (NLP), transformer models like BERT, RoBERTa, and DistilBERT have grown in popularity (NER). These models are built on a self-attention mechanism that enables them to focus on various input sequence elements and recognize intricate correlations between words.
\par
In most cases, the transformer architecture comprises of an encoder that receives a sequence of tokens as input and generates a sequence of hidden representations that can be applied to subsequent tasks like NER. For a particular downstream job, such as NER, the encoder is typically fine-tuned on a smaller labeled dataset after being pre-trained on a huge corpus of text using unsupervised learning.
\par
Transformer models for NER can be pre-trained and fine-tuned using a variety of techniques. Pre-training a multilingual transformer model on a sizable corpus of text from various languages, including Ukrainian, is one such method. A smaller labeled dataset of Ukrainian text can then be used to fine-tune the pre-trained model for NER.
\par
Another strategy is to use a sizable corpus of labeled data, such as the ** Ask **
\section{Metrics and procedures evaluation}


\chapter{Results}\label{cap:results}
\section{Analysis of experiment results}
\section{Discussion on impact of model architectures, pretraining and finetuning on the perfomance of the models}
\section{Comparing results to the current achievements}


\chapter{Conclusions}\label{cap:conclusions}
\section{Summary}
\section{Potential for the future, current limitations}

\bibliographystyle{plain}
\bibliography{mt.bib}
    
    
\end{document}